\documentclass[11pt, oneside]{book}
\usepackage{geometry}
\geometry{a4paper}

\usepackage{amsmath}
\usepackage{xfrac}

\usepackage{graphicx}
\usepackage{indentfirst}
\usepackage[colorlinks,urlcolor=blue]{hyperref}

\usepackage{listings}
\lstset{basicstyle=\footnotesize}

\usepackage{float}
\usepackage[usenames,dvipsnames]{color}

\usepackage{fontspec, xunicode, xltxtra}
\setmainfont{Hiragino Sans GB W3}
%\setmainfont{Microsoft YaHei}
\XeTeXlinebreaklocale "zh"
\XeTeXlinebreakskip = 0pt plus 1pt

\newcommand{\qhint}[1]{
\footnotesize
\vspace{0.2em}
\noindent
#1\par
\vspace{-0.5em}
\normalsize
}

\newcommand{\qsym}[1]{
\footnotesize
\noindent
#1\par
\normalsize
}

\newcommand{\qpara}[1]{
\vspace{0.3em}
\noindent
#1\par
\vspace{0.3em}
}

\newcommand{\qsamplelink}[1]{
\vspace{0.2em}
\noindent
#1\par
\vspace{0.1em}
}

\newcommand{\qurl}[1]{\footnotesize\url{#1}\normalsize}
\newcommand{\qqbox}[1]{\hspace{0.2em}\setlength{\fboxsep}{0.2em}\fcolorbox{black}{YellowGreen}{#1}\hspace{0.2em}}
\newcommand{\qtable}[1]{\footnotesize\vspace{0.5em}#1\vspace{0.5em}\normalsize}
\newcommand{\qsample}[1]{\hyperref[#1]{示例\ref*{#1}}}

\floatstyle{ruled}
\newfloat{sample}{H}{sample}[chapter]
\floatname{sample}{示例}

\setlength{\textheight}{680pt}

\title{七牛上传指南}
\author{梁涛 \\ @七牛云存储}
\date{2013-09-10}

\begin{document}

\maketitle
\tableofcontents

\chapter{术语}

\section{七牛云}

\qpara{所有对外放开的七牛服务及相关系统的统称,含云存储、云处理和云加速三大子系统。}
\qpara{本文通篇使用“七牛云”指代“七牛云存储子系统”这一实体。}

\section{业务服务器}

\qpara{七牛用户自建的、用于运行业务逻辑的服务器实体。}

\section{业务客户端/终端}

\qpara{七牛用户的用户(即最终用户)、用于运行交互逻辑的客户端实体。}

\section{高质量线路}

\qpara{机房(或IDC)之间使用的通信线路,具有低延迟(20~50毫秒)、极低丢包率、中间路由少或转发效率高、上下行带宽对等等特点。}
\qpara{注意:高速通信线路可能跨越不同运营商的异构网络。}

\section{低质量线路}

\qpara{机房(或IDC)与业务终端之间使用的通信线路,具有中等延迟(50~100毫秒)、低丢包率、偶发故障、偶发线路切换、上下行带宽不对等等特点。}

\section{终端接入环境}

\qpara{业务终端接入互联网的环境,常见的有LAN(局域网)、WLAN/Wi-Fi(无线局域网)、2G/3G(移动网)等。}
\qpara{注意:接入环境会随着终端物理位置随时发生改变,如离开房屋导致从Wi-Fi环境切换到3G环境,跨越3G信号塔覆盖区域导致在不同的接入基站之间切换。}

\section{资源}

\qpara{存入七牛云的数据体的统称,在有限前提下可与“文件”互换。}

\chapter{上传方式}

\section{直传}

\qpara{直传是最简单、最直观的上传方式,通过将单一资源封装在一次HTTP请求中传输至七牛云。}
\qpara{■\thinspace 优点:传输模型简单、API单一,同步操作。}
\qpara{■\thinspace 缺点:对大型资源不友好,失败概率与资源大小成正比,纠错成本高(只能重传)。}
\qpara{■\thinspace 适用场景:大小固定的微型/小型资源上传、服务器之间资源传输。}

\includegraphics[scale=2.7]{../pics/upload/one_put.jpg}

\section{断点续上传}

\qpara{通过将资源切分为固定大小的数据块(称为Block,4MB大小),封装在多次HTTP请求中逐一传输至七牛云。}
\qpara{可根据实际需求灵活地调整为串行传输或并发传输。}
\qpara{■\thinspace 优点:对大型资源友好,纠错成本小(只重传失败的数据块)。}
\qpara{■\thinspace 缺点:传输模型复杂、API多,异步操作。}
\qpara{■\thinspace 适用场景:大小固定的中型/大型资源上传,弱信号通信线路的上传,频繁切换接入点/线路的上传。}

\subsection{串行传输}

\qpara{断点续上传的串行传输与单一资源的直传类似,只是传输单元更小、纠错成本更小。相对地,多次HTTP请求会带来传输成本的上升。}
\qpara{■\thinspace 适用场景:出口(上行)带宽较小的环境,如家用ADSL接入线路。}

\includegraphics[scale=2.7]{../pics/upload/one_rput.jpg}

\qpara{虚线框:成功上传的数据范围。 \\ 实线框:可以读取的数据范围。}

\subsection{并发传输}

\qpara{断点续上传的并发传输与多资源的并发真传类似,但可以加快大型资源的传输速度。}
\qpara{■\thinspace 适用场景:出口(上行)带宽较大的环境,如商业专用接入线路。}

\includegraphics[scale=2.7]{../pics/upload/one_rputl.jpg}

\subsection{细粒度传输(Chunk)}

\qpara{断点续上传的细粒度传输构建在对数据块的进一步切分之上,也即将数据块(Block)切分为非固定大小的数据片(称为Chunk,小于4MB),更灵活地贴合传输环境的限制,甚至智能调整传输速率以提供更好的用户体验。每个数据片封装成一个HTTP请求,过小的数据片将带来额外的传输成本。}
\qpara{■\thinspace 适用场景:频繁切换接入点的环境,如地铁等信号弱、频繁重新接入网络的封闭空间。}

\includegraphics[scale=2.7]{../pics/upload/one_rput_chunk.jpg}

\section{追加上传(Append)}

\qpara{类似于操作系统打开文件时使用的Append参数,七牛云也提供相应语义的追加上传方式。}
\qpara{本方式同样采用切分数据、逐一上传的思路,但与断点续上传不同的是a)只切分成数据片(Chunk),b)每个数据片上传后即可立刻读取 。}
\qpara{■\thinspace 优点:对资源的最终大小无限制,可以实现流式传输。}
\qpara{■\thinspace 缺点:传输模型复杂,异步操作。}
\qpara{■\thinspace 适用场景:流式资源的上传,如监控音/视频流、直播视频流存储,长度无限制的日志。}

\includegraphics[scale=2.7]{../pics/upload/one_append.jpg}

\qpara{实线框:成功上传/可以读取的数据范围。}

\chapter{业务模型}

\qpara{待续……}

\chapter{授权}

\qpara{待续……}

\chapter{API}

\qpara{待续……}

\end{document}
