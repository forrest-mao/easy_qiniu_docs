\documentclass[11pt, oneside]{book}
\usepackage{geometry}
\geometry{a4paper}

\usepackage{amsmath}
\usepackage{xfrac}

\usepackage{graphicx}
\usepackage{indentfirst}
\usepackage[colorlinks,urlcolor=blue]{hyperref}

\usepackage{listings}
\lstset{basicstyle=\footnotesize}

\usepackage{float}
\usepackage[usenames,dvipsnames]{color}

\usepackage{fontspec, xunicode, xltxtra}
\setmainfont{Hiragino Sans GB W3}
%\setmainfont{Microsoft YaHei}
\XeTeXlinebreaklocale "zh"
\XeTeXlinebreakskip = 0pt plus 1pt

\newcommand{\qhint}[1]{
\footnotesize
\vspace{0.2em}
\noindent
#1\par
\vspace{-0.5em}
\normalsize
}

\newcommand{\qsym}[1]{
\footnotesize
\noindent
#1\par
\normalsize
}

\newcommand{\qpara}[1]{
\vspace{0.3em}
\noindent
#1\par
\vspace{0.3em}
}

\newcommand{\qsamplelink}[1]{
\vspace{0.2em}
\noindent
#1\par
\vspace{0.1em}
}

\newcommand{\qurl}[1]{\footnotesize\url{#1}\normalsize}
\newcommand{\qqbox}[1]{\hspace{0.2em}\setlength{\fboxsep}{0.2em}\fcolorbox{black}{YellowGreen}{#1}\hspace{0.2em}}
\newcommand{\qtable}[1]{\footnotesize\vspace{0.5em}#1\vspace{0.5em}\normalsize}
\newcommand{\qsample}[1]{\hyperref[#1]{示例\ref*{#1}}}

\floatstyle{ruled}
\newfloat{sample}{H}{sample}[chapter]
\floatname{sample}{示例}

\setlength{\textheight}{680pt}

\title{七牛云处理服务使用指南 \\ 图片处理部分}
\author{梁涛 \\ @七牛云存储}
\date{2013-08-19}

\begin{document}

\maketitle
\tableofcontents

\chapter{图片处理}

\section{获取图片信息}

\subsection{基本信息}

\qpara{在图片下载URL后附加\qqbox{imageInfo}指示符(区分大小写),可以获取JSON格式的图片基本信息,包括图片格式、宽高值(单位:像素)、色彩模型。}
\qpara{\textcolor{red}{注意:缩略图不支持该方法。}}

\begin{sample}
  \caption{获取图片基本信息}

    \qsamplelink{\qurl{http://newdocs.qiniudn.com/gogopher.jpg}}
    \qsym{$\Rightarrow$}
    \qurl{http://newdocs.qiniudn.com/gogopher.jpg?imageInfo}
    \linebreak

    \qhint{返回结果:}
\begin{lstlisting}
{
   "format":     "jpeg",
   "width":      640,     // 像素
   "height":     427,     // 像素
   "colorModel": "ycbcr"
}
\end{lstlisting}

  \label{imageInfo}
\end{sample}

\clearpage

\subsection{EXIF信息}

\qpara{EXIF(EXchangeable Image File Format),是专门为数码相机的照片设定的可交换图像文件格式,
详见\href{http://zh.wikipedia.org/wiki/EXIF}{维基百科}。}
\qpara{获取方式为在图片下载URL后附加\qqbox{exif}指示符(区分大小写)。}
\qpara{\textcolor{red}{注意:缩略图不支持该方法。}}

\begin{sample}
  \caption{获取图片EXIF信息}

    \qsamplelink{\qurl{http://newdocs.qiniudn.com/gogopher.jpg}}
    \qsym{$\Rightarrow$}
    \qurl{http://newdocs.qiniudn.com/gogopher.jpg?exif}
    \linebreak

    \qhint{返回结果:}
\begin{lstlisting}
{
   "DateTime" : {
      "type" : 2,
      "val" : "2011:11:19 17:09:23"
   },
   "ExposureBiasValue" : {
      "type" : 10,
      "val" : "0.33 EV"
   },
   "ExposureTime" : {
      "type" : 5,
      "val" : "1/50 sec."
   },
   "Model" : {
      "type" : 2,
      "val" : "Canon EOS 600D"
   },
   "ISOSpeedRatings" : {
      "type" : 3,
      "val" : "3200"
   },
   "ResolutionUnit" : {
      "type" : 3,
      "val" : " 英寸"
   },

   ......
}
\end{lstlisting}
  \label{exif}
\end{sample}

\clearpage

\section{生成缩略图}

\qpara{在图片下载URL后附加\qqbox{imageView}指示符(区分大小写)并指定缩略操作参数,可以生成该图片的缩略图。}
\qpara{调用接口:}
\begin{lstlisting}
<URL>?imageView/<Mode>/w/<Width>/h/<Height>/q/<Quality>/format/<Format>
\end{lstlisting}

%\begin{lstlisting}
%<URL>?imageView/<Mode>/w/<Width>/h/<Height>/q/<Quality>/format/<Format>
%     |________||_____||________||_________||__________||______________|
%          |       |        |         |           |            |
% 指示符 --+       |        |         |           |            |
%                  |        |         |           |            |
% 缩略模式 --------+        |         |           |            |
%                           |         |           |            |
% 缩略宽度(像素) ---------+         |           |            |
%                                     |           |            |
% 缩略高度(像素) -------------------+           |            |
%                                                 |            |
% 缩略质量(取值1-100) --------------------------+            |
%                                                              |
% 图片格式(支持jpg、gif、webp等) ----------------------------+
%\end{lstlisting}

\qpara{参数说明:}
\qtable{
\def\arraystretch{2}
\begin{tabular}{|l|l|l|}
\hline
名称 & 说明 & 备注\\
\hline
/\textless Mode\textgreater & 处理模式 & \\
\hline
/w/\textless Width\textgreater & 缩略宽度,单位:像素(px) & \\
\hline
/h/\textless Height\textgreater & 缩略高度,单位:像素(px) & \\
\hline
/q/\textless Quality\textgreater & 图片质量,取值范围1-100 & 可选,缺省为85 \\
\hline
/format/\textless Format\textgreater & 图片格式,支持jpg、gif、png、webp等 & 可选,缺省为原图格式 \\
\hline
\end{tabular}
}

\qpara{其中\textless mode\textgreater 分为如下4种情况:}
\qtable{
\def\arraystretch{2}
\begin{tabular}{|l|p{20em}|p{5em}|}
\hline
模式 & 说明 & 备注 \\
\hline
/1/w/\textless Width\textgreater /h/\textless Height\textgreater & 同时指定宽度和高度,等比裁剪原图正中部分并缩小为\textless width\textgreater x\textless height\textgreater 大小的缩略图。& \qsample{imageView-1-200x200} \\
\hline
/2/w/\textless Width\textgreater /h/\textless Height\textgreater & 同时指定宽度和高度,原图缩小为不超出\textless width\textgreater x\textless height\textgreater 大小的缩略图,避免裁剪长边。& \qsample{imageView-2-200x200} \\
\hline
/2/w/\textless Width\textgreater & 仅指定宽度,高度等比缩小。 & \qsample{imageView-2-200x200} \\
\hline
/2/h/\textless Height\textgreater & 仅指定高度,宽度等比缩小。 & \qsample{imageView-2-x200} \\
\hline
\end{tabular}
}

\clearpage

\begin{sample}
  \caption{裁剪正中部分,等比缩小生成200x200缩略图}
    \qsamplelink{\qurl{http://newdocs.qiniudn.com/gogopher.jpg}}
    \qsym{$\Rightarrow$}
    \qsamplelink{\qurl{http://newdocs.qiniudn.com/gogopher.jpg?imageView/1/w/200/h/200}}

    \begin{center}
      \includegraphics[scale=0.65]{../pics/gogopher_imageView_1_200x200.jpg}
    \end{center}
  \label{imageView-1-200x200}
\end{sample}

\begin{sample}
  \caption{宽度固定为200px,高度等比缩小,生成200x133缩略图}
    \qsamplelink{\qurl{http://newdocs.qiniudn.com/gogopher.jpg}}
    \qsym{$\Rightarrow$}
    \qsamplelink{\qurl{http://newdocs.qiniudn.com/gogopher.jpg?imageView/2/w/200/h/200}}
    \qsamplelink{\qurl{http://newdocs.qiniudn.com/gogopher.jpg?imageView/2/w/200}}

    \begin{center}
      \includegraphics[scale=0.65]{../pics/gogopher_imageView_2_200x200.jpg}
    \end{center}
  \label{imageView-2-200x200}
\end{sample}

\begin{sample}
  \caption{高度固定为200px,宽度等比缩小,生成300x200缩略图}
    \qsamplelink{\qurl{http://newdocs.qiniudn.com/gogopher.jpg}}
    \qsym{$\Rightarrow$}
    \qsamplelink{\qurl{http://newdocs.qiniudn.com/gogopher.jpg?imageView/2/h/200}}

    \begin{center}
      \includegraphics[scale=0.65]{../pics/gogopher_imageView_2_x200.jpg}
    \end{center}
  \label{imageView-2-x200}
\end{sample}

\clearpage

\section{高级图片处理}

\qpara{除了能方便地生成缩略图外,七牛云处理服务还提供其它高级图片处理功能,包括缩略、裁剪、旋转等等。}
\qpara{调用接口:}
\begin{lstlisting}
<URL>?imageMogr/v2/auto-orient
                  /thumbnail/<ImageSize>
                  /gravity/<GravityType>
                  /crop/<ImageSizeAndOffset>
                  /quality/<Quality>
                  /rotate/<Degree>
                  /format/<Format>
\end{lstlisting}

\qpara{参数说明:}
\qtable{
\def\arraystretch{2}
\begin{tabular}{|l|p{13em}|p{10em}|}
\hline
名称 & 说明 & 备注\\
\hline
/auto-orient & 根据原图EXIF信息自动旋正,便于后续处理 & 可选,建议放在首位 \\
\hline
/thumbnail/\textless ImageSize\textgreater & 参看\hyperref[thumbnail-spec]{缩略操作参数表} & 可选,缺省为不缩略 \\
\hline
/gravity/\textless GravityType\textgreater & 参看\hyperref[crop-anchor-spec]{裁剪锚点参数表},只会影响其后的裁剪偏移参数 & 可选,缺省为左上角(NorthWest) \\
\hline
/crop/\textless ImageSizeAndOffset\textgreater & 参看\hyperref[crop-spec]{裁剪操作参数表} & 可选,缺省为不裁剪 \\
\hline
/quality/\textless Quality\textgreater & 图片质量,取值范围1-100 & 可选,缺省为85 \\
\hline
/rotate/\textless Degree\textgreater & 旋转角度,取值范围1-360 & 可选,缺省为不旋转 \\
\hline
/format/\textless Format\textgreater & 图片格式,支持jpg、gif、png、webp等 & 可选,缺省为原图格式 \\
\hline
\end{tabular}
}

\clearpage

\subsection{缩略}

\qpara{缩略操作参数表:}
\qtable{
\label{thumbnail-spec}
\begin{tabular}[t]{|l|p{18em}|p{5em}|}
\hline
参数名称 & 说明 & 示例 \\
\hline
/thumbnail/!\textless Scale\textgreater p & 基于原图大小,按指定百分比缩放。取值范围0-1000 & \qsample{imageMogr-thumbnail-75p} \\
\hline
/thumbnail/!\textless Scale\textgreater px & 以百分比形式指定目标图片宽度,高度等比缩放。取值范围0-1000 & \qsample{imageMogr-thumbnail-75p} \\
\hline
/thumbnail/!x\textless Scale\textgreater p & 以百分比形式指定目标图片高度,宽度等比缩放。取值范围0-1000 & \qsample{imageMogr-thumbnail-75p} \\
\hline
/thumbnail/\textless Width\textgreater x & 指定目标图片宽度,高度等比缩放。取值范围0-10000 & \qsample{imageMogr-thumbnail-700px} \\
\hline
/thumbnail/x\textless Height\textgreater & 指定目标图片高度,宽度等比缩放。取值范围0-10000 & \qsample{imageMogr-thumbnail-700px} \\
\hline
/thumbnail/\textless Width\textgreater x\textless Height\textgreater & 限定长边,短边自适应缩放,将目标图片限制在指定宽高矩形内。取值范围0-10000 & \qsample{imageMogr-thumbnail-300x300} \\
\hline
/thumbnail/!\textless Width\textgreater x\textless Height\textgreater r & 限定短边,长边自适应缩放,目标图片会延伸至指定宽高矩形外。取值范围0-10000 & \qsample{imageMogr-thumbnail-200x200r} \\
\hline
/thumbnail/\textless Width\textgreater x\textless Height\textgreater ! & 限定目标图片宽高值,忽略原图宽高比例,按照指定宽高值强行缩略,可能导致目标图片变形。取值范围0-10000 & \qsample{imageMogr-thumbnail-200x300-force} \\
\hline
/thumbnail/\textless Width\textgreater x\textless Height\textgreater \textgreater & 当原图尺寸大于给定的宽度或高度时,按照给定宽高值缩小。取值范围0-10000 & \qsample{imageMogr-thumbnail-200x300-greater} \\
\hline
/thumbnail/\textless Width\textgreater x\textless Height\textgreater \textless & 当原图尺寸小于给定的宽度或高度时,按照给定宽高值放大。取值范围0-10000 &  \qsample{imageMogr-thumbnail-700x600-less} \\
\hline
/thumbnail/\textless Area\textgreater @ & 按原图高宽比例等比缩放,缩放后的像素数量不超过指定值。取值范围0-10$^8$ & \qsample{imageMogr-thumbnail-350000-at} \\
\hline
\end{tabular}
}

\begin{sample}
  \caption{生成480x320缩略图}
    \qsamplelink{\qurl{http://newdocs.qiniudn.com/gogopher.jpg}}
    \qsym{$\Rightarrow$}
    \qhint{等比缩小75\%}
    \qsamplelink{\qurl{http://newdocs.qiniudn.com/gogopher.jpg?imageMogr/v2/thumbnail/!75p}}
    \qhint{按原宽度75\%等比缩小}
    \qsamplelink{\qurl{http://newdocs.qiniudn.com/gogopher.jpg?imageMogr/v2/thumbnail/!75px}}
    \qhint{按原高度75\%等比缩小}
    \qsamplelink{\qurl{http://newdocs.qiniudn.com/gogopher.jpg?imageMogr/v2/thumbnail/!x75p}}

    \begin{center}
      \includegraphics[scale=0.65]{../pics/gogopher_imageMogr_75p.jpg}
    \end{center}
  \label{imageMogr-thumbnail-75p}
\end{sample}

\begin{sample}
  \caption{生成700x467放大图}
    \qsamplelink{\qurl{http://newdocs.qiniudn.com/gogopher.jpg}}
    \qsym{$\Rightarrow$}
    \qhint{指定新宽度为700px}
    \qsamplelink{\qurl{http://newdocs.qiniudn.com/gogopher.jpg?imageMogr/v2/thumbnail/700x}}
    \qhint{指定新高度为467px}
    \qsamplelink{\qurl{http://newdocs.qiniudn.com/gogopher.jpg?imageMogr/v2/thumbnail/x467}}

    \begin{center}
      \includegraphics[scale=0.65]{../pics/gogopher_imageMogr_700px.jpg}
    \end{center}
  \label{imageMogr-thumbnail-700px}
\end{sample}

\begin{sample}
  \caption{限定长边,生成不超过300x300的缩略图}
    \qsamplelink{\qurl{http://newdocs.qiniudn.com/gogopher.jpg}}
    \qsym{$\Rightarrow$}
    \qsamplelink{\qurl{http://newdocs.qiniudn.com/gogopher.jpg?imageMogr/v2/thumbnail/300x300}}

    \begin{center}
      \includegraphics[scale=0.65]{../pics/gogopher_imageMogr_thumbnail_300x300.jpg}
    \end{center}
  \label{imageMogr-thumbnail-300x300}
\end{sample}

\begin{sample}
  \caption{限定短边,生成不小于200x200的缩略图}
    \qsamplelink{\qurl{http://newdocs.qiniudn.com/gogopher.jpg}}
    \qsym{$\Rightarrow$}
    \qsamplelink{\qurl{http://newdocs.qiniudn.com/gogopher.jpg?imageMogr/v2/thumbnail/!200x200r}}

    \begin{center}
      \includegraphics[scale=0.65]{../pics/gogopher_imageMogr_thumbnail_200x200r.jpg}
    \end{center}
  \label{imageMogr-thumbnail-200x200r}
\end{sample}

\begin{sample}
  \caption{强制生成200x300的缩略图}
    \qsamplelink{\qurl{http://newdocs.qiniudn.com/gogopher.jpg}}
    \qsym{$\Rightarrow$}
    \qsamplelink{\qurl{http://newdocs.qiniudn.com/gogopher.jpg?imageMogr/v2/thumbnail/200x300!}}

    \begin{center}
      \includegraphics[scale=0.65]{../pics/gogopher_imageMogr_thumbnail_200x300_force.jpg}
    \end{center}
  \label{imageMogr-thumbnail-200x300-force}
\end{sample}

\begin{sample}
  \caption{原图大于指定长宽矩形,按长边自动缩小为200x133缩略图}
    \qsamplelink{\qurl{http://newdocs.qiniudn.com/gogopher.jpg}}
    \qsym{$\Rightarrow$}
    \qsamplelink{\qurl{http://newdocs.qiniudn.com/gogopher.jpg?imageMogr/v2/thumbnail/200x300>}}

    \begin{center}
      \includegraphics[scale=0.65]{../pics/gogopher_imageMogr_thumbnail_200x300_greater.jpg}
    \end{center}
  \label{imageMogr-thumbnail-200x300-greater}
\end{sample}

\begin{sample}
  \caption{原图小于指定长宽矩形,按长边自动拉伸为700x467放大图}
    \qsamplelink{\qurl{http://newdocs.qiniudn.com/gogopher.jpg}}
    \qsym{$\Rightarrow$}
    \qsamplelink{\qurl{http://newdocs.qiniudn.com/gogopher.jpg?imageMogr/v2/thumbnail/700x600<}}

    \begin{center}
      \includegraphics[scale=0.65]{../pics/gogopher_imageMogr_thumbnail_700x600_less.jpg}
    \end{center}
  \label{imageMogr-thumbnail-700x600-less}
\end{sample}

\begin{sample}
  \caption{生成图的像素总数小于指定值}
    \qsamplelink{\qurl{http://newdocs.qiniudn.com/gogopher.jpg}}
    \qsym{$\Rightarrow$}
    \qsamplelink{\qurl{http://newdocs.qiniudn.com/gogopher.jpg?imageMogr/v2/thumbnail/350000@}}

    \begin{center}
      \includegraphics[scale=0.65]{../pics/gogopher_imageMogr_thumbnail_350000_at.jpg}
    \end{center}
  \label{imageMogr-thumbnail-350000-at}
\end{sample}

\clearpage

\subsection{裁剪}

\qpara{裁剪操作参数表:}
\qtable{
\label{crop-spec}
\begin{tabular}[t]{|l|p{20em}|p{5em}|}
\hline
参数名称 & 说明 & 示例 \\
\hline
/crop/\textless Width\textgreater x & 指定目标图片宽度,高度不变。取值范围0-10000 & \qsample{imageMogr-crop-300x} \\
\hline
/crop/x\textless Height\textgreater & 指定目标图片高度,宽度不变。取值范围0-10000 & \qsample{imageMogr-crop-x200} \\
\hline
/crop/\textless Width\textgreater x\textless Height\textgreater & 同时指定目标图片宽高。取值范围0-10000 & \qsample{imageMogr-crop-300x300} \\
\hline
\end{tabular}
}

\begin{sample}
  \caption{生成300x427裁剪图}
    \qsamplelink{\qurl{http://newdocs.qiniudn.com/gogopher.jpg}}
    \qsym{$\Rightarrow$}
    \qsamplelink{\qurl{http://newdocs.qiniudn.com/gogopher.jpg?imageMogr/v2/crop/300x}}

    \begin{center}
      \includegraphics[scale=0.65]{../pics/gogopher_imageMogr_crop_300x.jpg}
    \end{center}
  \label{imageMogr-crop-300x}
\end{sample}

\begin{sample}
  \caption{生成640x200裁剪图}
    \qsamplelink{\qurl{http://newdocs.qiniudn.com/gogopher.jpg}}
    \qsym{$\Rightarrow$}
    \qsamplelink{\qurl{http://newdocs.qiniudn.com/gogopher.jpg?imageMogr/v2/crop/x200}}

    \begin{center}
      \includegraphics[scale=0.65]{../pics/gogopher_imageMogr_crop_x200.jpg}
    \end{center}
  \label{imageMogr-crop-x200}
\end{sample}

\begin{sample}
  \caption{生成300x300裁剪图}
    \qsamplelink{\qurl{http://newdocs.qiniudn.com/gogopher.jpg}}
    \qsym{$\Rightarrow$}
    \qsamplelink{\qurl{http://newdocs.qiniudn.com/gogopher.jpg?imageMogr/v2/crop/300x300}}

    \begin{center}
      \includegraphics[scale=0.65]{../pics/gogopher_imageMogr_crop_300x300.jpg}
    \end{center}
  \label{imageMogr-crop-300x300}
\end{sample}

\clearpage

\qpara{裁剪偏移参数表:}
\qtable{
\label{crop-offset-spec}
\begin{tabular}[t]{|l|p{16em}|p{5em}|}
\hline
参数名称 & 说明 & 示例 \\
\hline
/crop/!\{size\}a\textless X-offset\textgreater a\textless Y-offset\textgreater & 相对于偏移锚点,向右偏移\textless X-offset\textgreater 个像素,同时向下偏移\textless Y-offset\textgreater 个像素,取值范围0-1000 & \qsample{imageMogr-crop-300x300-a30a100} \\
\hline
/crop/!\{size\}-\textless X-offset\textgreater a\textless Y-offset\textgreater & 相对于偏移锚点,向下偏移\textless Y-offset\textgreater 个像素,同时从指定宽度中减去\textless X-offset\textgreater 个像素,取值范围0-1000 & \qsample{imageMogr-crop-300x300-a30s100} \\
\hline
/crop/!\{size\}a\textless X-offset\textgreater -\textless Y-offset\textgreater & 相对于偏移锚点,向右偏移\textless X-offset\textgreater 个像素,同时从指定高度中减去\textless Y-offset\textgreater 个像素,取值范围0-1000 & \qsample{imageMogr-crop-300x300-s30a100} \\
\hline
/crop/!\{size\}-\textless X-offset\textgreater -\textless Y-offset\textgreater & 相对于偏移锚点,从指定宽度中减去\textless X-offset\textgreater 个像素,同时从指定高度中减去\textless Y-offset\textgreater 个像素,取值范围0-1000 & \qsample{imageMogr-crop-300x300-s30s100} \\
\hline
\end{tabular}
}

\qpara{\textcolor{red}{注意1:必须同时指定横轴偏移和纵轴偏移。}}
\qpara{\textcolor{red}{注意2:计算偏移值会受到位置偏移指示符(/gravity/\textless GravityType\textgreater)影响。}默认为相对于左上角计算偏移值(即NorthWest),参看下表。}

\qpara{裁剪锚点参数表(不区分大小写):}
\qtable{
\label{crop-anchor-spec}
\begin{tabular}[t]{|l|c|r|}
\hline
\parbox[t][6em]{6em}{NorthWest \\ \qsample{imageMogr-crop-northwest}} & \parbox[t][6em]{6em}{\makebox[6em][c]{North} \\ \makebox[6em][c]{\qsample{imageMogr-crop-north}}} & \parbox[t][6em]{6em}{\makebox[6em][r]{NorthEast} \\ \makebox[6em][r]{\qsample{imageMogr-crop-northeast}}} \\
\hline
\parbox[c][7em]{6em}{West \\ \qsample{imageMogr-crop-west}} & \parbox[c][7em]{6em}{\makebox[6em][c]{Center} \\ \makebox[6em][c]{\qsample{imageMogr-crop-center}}} & \parbox[c][7em]{6em}{\makebox[6em][r]{East} \\ \makebox[6em][r]{\qsample{imageMogr-crop-east}}} \\
\hline
\parbox[b][6em]{6em}{SouthWest \\ \qsample{imageMogr-crop-southwest}} & \parbox[b][6em]{6em}{\makebox[6em][c]{South} \\ \makebox[6em][c]{\qsample{imageMogr-crop-south}}} & \parbox[b][6em]{6em}{\makebox[6em][r]{SouthEast} \\ \makebox[6em][r]{\qsample{imageMogr-crop-southeast}}} \\
\hline
\end{tabular}
}

\begin{sample}
  \caption{生成300x300裁剪图,偏移距离30x100}
    \qsamplelink{\qurl{http://newdocs.qiniudn.com/gogopher.jpg}}
    \qsym{$\Rightarrow$}
    \qsamplelink{\qurl{http://newdocs.qiniudn.com/gogopher.jpg?imageMogr/v2/crop/!300x300a30a100}}

    \begin{center}
      \includegraphics[scale=0.65]{../pics/gogopher_imageMogr_crop_300x300_a30a100.jpg}
    \end{center}
  \label{imageMogr-crop-300x300-a30a100}
\end{sample}

\begin{sample}
  \caption{生成300x200裁剪图,偏移距离30x0}
    \qsamplelink{\qurl{http://newdocs.qiniudn.com/gogopher.jpg}}
    \qsym{$\Rightarrow$}
    \qsamplelink{\qurl{http://newdocs.qiniudn.com/gogopher.jpg?imageMogr/v2/crop/!300x300a30-100}}

    \begin{center}
      \includegraphics[scale=0.65]{../pics/gogopher_imageMogr_crop_300x300_a30s100.jpg}
    \end{center}
  \label{imageMogr-crop-300x300-a30s100}
\end{sample}

\begin{sample}
  \caption{生成270x300裁剪图,偏移距离0x100}
    \qsamplelink{\qurl{http://newdocs.qiniudn.com/gogopher.jpg}}
    \qsym{$\Rightarrow$}
    \qsamplelink{\qurl{http://newdocs.qiniudn.com/gogopher.jpg?imageMogr/v2/crop/!300x300-30a100}}

    \begin{center}
      \includegraphics[scale=0.65]{../pics/gogopher_imageMogr_crop_300x300_s30a100.jpg}
    \end{center}
  \label{imageMogr-crop-300x300-s30a100}
\end{sample}

\begin{sample}
  \caption{生成270x200裁剪图,偏移距离0x0}
    \qsamplelink{\qurl{http://newdocs.qiniudn.com/gogopher.jpg}}
    \qsym{$\Rightarrow$}
    \qsamplelink{\qurl{http://newdocs.qiniudn.com/gogopher.jpg?imageMogr/v2/crop/!300x300-30-100}}

    \begin{center}
      \includegraphics[scale=0.65]{../pics/gogopher_imageMogr_crop_300x300_s30s100.jpg}
    \end{center}
  \label{imageMogr-crop-300x300-s30s100}
\end{sample}

\begin{sample}
  \caption{锚点在左上角(NorthWest),生成300x300缩略图}
    \qsamplelink{\qurl{http://newdocs.qiniudn.com/gogopher.jpg}}
    \qsym{$\Rightarrow$}
    \qsamplelink{\qurl{http://newdocs.qiniudn.com/gogopher.jpg?imageMogr/v2/gravity/NorthWest/crop/300x300}}

    \begin{center}
      \includegraphics[scale=0.65]{../pics/gogopher_imageMogr_crop_northwest.jpg}
    \end{center}
  \label{imageMogr-crop-northwest}
\end{sample}

\begin{sample}
  \caption{锚点在正上方(North),生成300x300缩略图}
    \qsamplelink{\qurl{http://newdocs.qiniudn.com/gogopher.jpg}}
    \qsym{$\Rightarrow$}
    \qsamplelink{\qurl{http://newdocs.qiniudn.com/gogopher.jpg?imageMogr/v2/gravity/North/crop/300x300}}

    \begin{center}
      \includegraphics[scale=0.65]{../pics/gogopher_imageMogr_crop_north.jpg}
    \end{center}
  \label{imageMogr-crop-north}
\end{sample}

\begin{sample}
  \caption{锚点在右上角(NorthEast),生成300x300缩略图}
    \qsamplelink{\qurl{http://newdocs.qiniudn.com/gogopher.jpg}}
    \qsym{$\Rightarrow$}
    \qsamplelink{\qurl{http://newdocs.qiniudn.com/gogopher.jpg?imageMogr/v2/gravity/NorthEast/crop/300x300}}

    \begin{center}
      \includegraphics[scale=0.65]{../pics/gogopher_imageMogr_crop_northeast.jpg}
    \end{center}
  \label{imageMogr-crop-northeast}
\end{sample}

\begin{sample}
  \caption{锚点在正左方(West),生成300x300缩略图}
    \qsamplelink{\qurl{http://newdocs.qiniudn.com/gogopher.jpg}}
    \qsym{$\Rightarrow$}
    \qsamplelink{\qurl{http://newdocs.qiniudn.com/gogopher.jpg?imageMogr/v2/gravity/West/crop/300x300}}

    \begin{center}
      \includegraphics[scale=0.65]{../pics/gogopher_imageMogr_crop_west.jpg}
    \end{center}
  \label{imageMogr-crop-west}
\end{sample}

\begin{sample}
  \caption{锚点在正中(Center),生成300x300缩略图}
    \qsamplelink{\qurl{http://newdocs.qiniudn.com/gogopher.jpg}}
    \qsym{$\Rightarrow$}
    \qsamplelink{\qurl{http://newdocs.qiniudn.com/gogopher.jpg?imageMogr/v2/gravity/Center/crop/300x300}}

    \begin{center}
      \includegraphics[scale=0.65]{../pics/gogopher_imageMogr_crop_center.jpg}
    \end{center}
  \label{imageMogr-crop-center}
\end{sample}

\begin{sample}
  \caption{锚点在正右方(East),生成300x300缩略图}
    \qsamplelink{\qurl{http://newdocs.qiniudn.com/gogopher.jpg}}
    \qsym{$\Rightarrow$}
    \qsamplelink{\qurl{http://newdocs.qiniudn.com/gogopher.jpg?imageMogr/v2/gravity/East/crop/300x300}}

    \begin{center}
      \includegraphics[scale=0.65]{../pics/gogopher_imageMogr_crop_east.jpg}
    \end{center}
  \label{imageMogr-crop-east}
\end{sample}

\begin{sample}
  \caption{锚点在左下角(SouthWest),生成300x300缩略图}
    \qsamplelink{\qurl{http://newdocs.qiniudn.com/gogopher.jpg}}
    \qsym{$\Rightarrow$}
    \qsamplelink{\qurl{http://newdocs.qiniudn.com/gogopher.jpg?imageMogr/v2/gravity/SouthWest/crop/300x300}}

    \begin{center}
      \includegraphics[scale=0.65]{../pics/gogopher_imageMogr_crop_southwest.jpg}
    \end{center}
  \label{imageMogr-crop-southwest}
\end{sample}

\begin{sample}
  \caption{锚点在正下方(South),生成300x300缩略图}
    \qsamplelink{\qurl{http://newdocs.qiniudn.com/gogopher.jpg}}
    \qsym{$\Rightarrow$}
    \qsamplelink{\qurl{http://newdocs.qiniudn.com/gogopher.jpg?imageMogr/v2/gravity/South/crop/300x300}}

    \begin{center}
      \includegraphics[scale=0.65]{../pics/gogopher_imageMogr_crop_south.jpg}
    \end{center}
  \label{imageMogr-crop-south}
\end{sample}

\begin{sample}
  \caption{锚点在右下角(SouthEast),生成300x300缩略图}
    \qsamplelink{\qurl{http://newdocs.qiniudn.com/gogopher.jpg}}
    \qsym{$\Rightarrow$}
    \qsamplelink{\qurl{http://newdocs.qiniudn.com/gogopher.jpg?imageMogr/v2/gravity/SouthEast/crop/300x300}}

    \begin{center}
      \includegraphics[scale=0.65]{../pics/gogopher_imageMogr_crop_southeast.jpg}
    \end{center}
  \label{imageMogr-crop-southeast}
\end{sample}

\clearpage

\subsection{旋转}

\begin{sample}
  \caption{顺时针旋转45度}
    \qsamplelink{\qurl{http://newdocs.qiniudn.com/gogopher.jpg}}
    \qsym{$\Rightarrow$}
    \qsamplelink{\qurl{http://newdocs.qiniudn.com/gogopher.jpg?imageMogr/v2/rotate/45}}

    \begin{center}
      \includegraphics[scale=0.65]{../pics/gogopher_imageMogr_rotate_45.jpg}
    \end{center}
  \label{imageMogr-rotate-45}
\end{sample}

\clearpage

\subsection{组合处理}

\qpara{高级图片处理支持组合多个操作,即上一个操作的输出作为下一个操作的输入。}
\qpara{操作指示符在URL中的出现顺序决定组合处理的顺序,详见以下示例。}

\begin{sample}
  \caption{缩小后裁剪}
    \qsamplelink{\qurl{http://newdocs.qiniudn.com/gogopher.jpg}}
    \qsym{$\Rightarrow$}
    \qsamplelink{\qurl{http://newdocs.qiniudn.com/gogopher.jpg?imageMogr/v2/thumbnail/400x200!/gravity/center/crop/100x100}}

    \begin{center}
      \includegraphics[scale=0.65]{../pics/gogopher_imageMogr_thumbnail_crop_100x100.jpg}
    \end{center}
  \label{imageMogr-thumbnail-crop-100x100}
\end{sample}

\begin{sample}
  \caption{裁剪后放大}
    \qsamplelink{\qurl{http://newdocs.qiniudn.com/gogopher.jpg}}
    \qsym{$\Rightarrow$}
    \qsamplelink{\qurl{http://newdocs.qiniudn.com/gogopher.jpg?imageMogr/v2/gravity/center/crop/100x100/thumbnail/200x200<}}

    \begin{center}
      \includegraphics[scale=0.65]{../pics/gogopher_imageMogr_crop_thumbnail_200x200.jpg}
    \end{center}
  \label{imageMogr-crop-thumbnail-200x200}
\end{sample}

\clearpage

\section{添加水印}

\qpara{七牛云处理服务支持添加图片水印或文字水印,使用方法为在下载URL后添加\qqbox{watermark}指示符(区分大小写)并附加相关参数。}
\qpara{图片水印调用接口:}
\begin{lstlisting}
<URL>?watermark/1
               /image/<EncodedImageURL>
               /dissolve/<Dissolve>
               /gravity/<Gravity>
               /dx/<DistanceX>
               /dy/<DistanceY>
\end{lstlisting}
\qpara{文字水印调用接口:}
\begin{lstlisting}
<URL>?watermark/2
               /text/<EncodedText>
               /font/<EncodedFontName>
               /fontsize/<FontSize>
               /fill/<EncodedTextColor>
               /dissolve/<Dissolve>
               /gravity/<Gravity>
               /dx/<DistanceX>
               /dy/<DistanceY>
\end{lstlisting}

\qpara{水印操作参数说明表:}
\qtable{
\begin{tabular}[t]{|l|p{14em}|p{10em}|}
\hline
参数名称 & 说明 & 示例 \\
\hline
/image/\textless EncodedImageURL\textgreater & 水印源图片,必须为有效网址 & 需\href{http://zh.wikipedia.org/zh-cn/Base64}{Base64编码} \\
\hline
/text/\textless EncodedText\textgreater & 水印文字 & 需Base64编码 \\
\hline
/font/\textless EncodedFontName\textgreater & 水印文字字体 & 需Base64编码,使用非英文文字时必填 \\
\hline
/fontsize/\textless FontSize\textgreater & 水印文字字号,单位: 缇(\href{http://en.wikipedia.org/wiki/Twip}{twip}),等于$\sfrac{1}{20}$磅(pt) & 可选,缺省为0(默认字号) \\
\hline
/fill/\textless EncodedTextColor\textgreater & 水印文字颜色,遵循RGB色彩模型,可为颜色名称或\#RRGGBB格式(十六进制),参考\href{http://www.rapidtables.com/web/color/RGB_Color.htm}{RGB颜色编码表} & 需Base64编码 \\
\hline
/dissolve/\textless Dissolve\textgreater & 透明度,取值范围1-100 & 可选,缺省值为100(完全不透明) \\
\hline
/gravity/\textless Gravity\textgreater & 水印位置,参考水印位置参数表 & 可选,缺省为右下角(SouthEast) \\
\hline
/dx/\textless DistanceX\textgreater & 横轴边距,单位:像素(px) & 可选,缺省值为10 \\
\hline
/dy/\textless DistanceY\textgreater & 纵轴边距,单位:像素(px) & 可选,缺省值为10 \\
\hline
\end{tabular}
}

\begin{sample}
  \caption{图片水印}
    \qsamplelink{\qurl{http://newdocs.qiniudn.com/gogopher.jpg}}
    \qsym{$\Rightarrow$}
    \qsamplelink{\ttfamily\footnotesize\href{http://newdocs.qiniudn.com/gogopher.jpg?watermark/1/image/aHR0cDovL3d3dy5iMS5xaW5pdWRuLmNvbS9pbWFnZXMvbG9nby0yLnBuZw==}{http://newdocs.qiniudn.com/gogopher.jpg?watermark/1/image/aHR0cDov...}\normalsize}

    \begin{center}
      \includegraphics[scale=0.69]{../pics/watermark_pic.jpg}
    \end{center}
\end{sample}

\begin{sample}
  \caption{文字水印}
    \qsamplelink{\qurl{http://newdocs.qiniudn.com/gogopher.jpg}}
    \qsym{$\Rightarrow$}
    \qurl{http://newdocs.qiniudn.com/gogopher.jpg?watermark/2/text/5LiD54mb5LqR5a2Y5YKo/font/U2ltSGVp/fontsize/600/fill/Izc0RDJGRg==/gravity/North}
    \linebreak

\footnotesize
\qpara{字体 SimHei \\ 字号 600twips (30pt) \\ 位置 正上方 \\ 颜色 \#74D2FF}
\normalsize

    \begin{center}
      \includegraphics[scale=0.69]{../pics/watermark_text.jpg}
    \end{center}
\end{sample}

\clearpage

\section{设定处理规格别名(友好URL)}

\qpara{七牛云服务支持为图片处理规格设定别名,避免往下载URL中引入QueryString字符串(即问号后的部分),从而规避可能引发的各种使用不便。}
\qpara{每个资源空间(Bucket)可以设定多个别名和分隔符,以提供灵活的使用方式。}
\qpara{七牛\href{https://portal.qiniu.com}{开发者后台}或\href{http://docs.qiniu.com/tools/v6/qboxrsctl.html}{开发者工具(qboxrsctl)}来都可以定制别名。前文描述的各种处理均支持使用别名。}

\qpara{调用接口:}
\begin{lstlisting}
http://<Domain>/<Key><Separator><Alias>
       |______| |___||_________||_____|
          |       |       |        |
          |       |       |        +-- 规格别名
          |       |       |
          |       |       +----------- 分隔符
          |       |
          |       +------------------- 资源名
          |
          +--------------------------- 域名
\end{lstlisting}

\begin{sample}
  \caption{使用qboxrsctl定制缩略图别名}
    \begin{lstlisting}
qboxrsctl login <email> <password>                         # 登录账号
qboxrsctl style <bucket> small.jpg imageView/1/w/200/h/200 # 设定规格别名
qboxrsctl separator <bucket> -                             # 设定分隔符
    \end{lstlisting}

    \linebreak
    \qsamplelink{\qurl{http://newdocs.qiniudn.com/gogopher.jpg?imageView/1/w/200/h/200}}
    \qsym{$=$}
    \qurl{http://newdocs.qiniudn.com/gogopher.jpg-small.jpg}
    \linebreak

\qhint{其中,各组成部分分别为:}
\begin{lstlisting}
http://newdocs.qiniudn.com/gogopher.jpg-small.jpg
       |_________________| |__________|||_______|
                |                |     |    |
                |                |     |    +-- 规格别名
                |                |     |    
                |                |     +------- 分隔符
                |                |
                |                +------------- 资源名
                |
                +------------------------------ 域名
\end{lstlisting}
\end{sample}

\clearpage

\section{图片预转}

\qpara{通常图片处理是在下载时执行,但可能增加下载延迟(如图片很大,处理时间会变长)。为避免增加无意义的下载延迟,可以使用预转服务,在图片上传成功后即刻处理。}
\qpara{预转属于异步操作,上传成功完成后自动触发服务器处理,无须等待其完成,进一步提升使用体验。}
\qpara{使用方法请参考\href{http://docs.qiniu.com/api/v6/put.html\#uploadToken-asyncOps}{uploadToken的asyncOps字段。}}

\clearpage

\section{FAQ}

\begin{itemize}
  \item 图片处理执行时机?\par
  \qhint{通常而言,图片处理是在下载时执行,但可以使用预转功能将执行时机提前到图片上传成功时。}

  \item 图片处理结果是否计入存储用量?\par
  \qhint{图片处理结果将被缓存,不计入存储用量中。}
\end{itemize}

\end{document}
