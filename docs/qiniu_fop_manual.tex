\documentclass[11pt, oneside]{book}
\usepackage{geometry}
\geometry{a4paper}

\usepackage{CJKutf8}

\usepackage{graphicx}
\usepackage{indentfirst}
\usepackage[colorlinks,urlcolor=blue]{hyperref}

\usepackage{listings}
\lstset{basicstyle=\ttfamily}

\usepackage{float}
\usepackage{color}

\newcommand{\qpar}[1]{
\vspace{0.25em}
\noindent
#1\par
\vspace{0.25em}
}

\newcommand{\qurl}[1]{\url{#1}}
\newcommand{\qqbox}[1]{\hspace{0.2em}\fbox{#1}\hspace{0.2em}}
\newcommand{\qtable}[1]{\footnotesize\vspace{0.5em}#1\vspace{0.5em}\normalsize}

\title{七牛云处理服务使用指南}
\author{梁涛 \\ @七牛云存储}
\date{2013-08-19}

\begin{document}

\begin{CJK*}{UTF8}{song}
\maketitle

\floatstyle{ruled}
\newfloat{sample}{h}{sample}[chapter]
\floatname{sample}{示例}

\chapter{图片处理}

\section{获取图片信息}

\subsection{基本信息}

\qpar{在图片下载URL后附加\qqbox{imageInfo}指示符(区分大小写),可以获取JSON格式的图片基本信息,包括图片格式、宽高值(单位:像素)、色彩模型。\textcolor{red}{注意:缩略图不支持该方法。}}

\begin{sample}
  \caption{获取图片基本信息}

    \qpar{\qurl{http://qiniuphotos.qiniudn.com/gogopher.jpg}}
    \qpar{$\Rightarrow$}
    \qpar{\qurl{http://qiniuphotos.qiniudn.com/gogopher.jpg?imageInfo}}

    \qpar{返回结果:}
\begin{verbatim}
{
   "format":     "jpeg",
   "width":      640,     // 像素
   "height":     427,     // 像素
   "colorModel": "ycbcr"
}
\end{verbatim}

  \label{imageInfo}
\end{sample}

\subsection{EXIF信息}

\qpar{EXIF(EXchangeable Image File Format),是专门为数码相机的照片设定的可交换图像文件格式,
详见\href{http://zh.wikipedia.org/wiki/EXIF}{维基百科}。}
\qpar{获取方式为在图片下载URL后附加\qqbox{exif}指示符(区分大小写)。\textcolor{red}{注意:缩略图不支持该方法。}}

\begin{sample}
  \caption{获取图片EXIF信息}

    \qpar{\qurl{http://qiniuphotos.qiniudn.com/gogopher.jpg}}
    \qpar{$\Rightarrow$}
    \qpar{\qurl{http://qiniuphotos.qiniudn.com/gogopher.jpg?exif}}

    \qpar{返回结果:}
\begin{verbatim}
{
   "DateTime" : {
      "type" : 2,
      "val" : "2011:11:19 17:09:23"
   },
   "ExposureBiasValue" : {
      "type" : 10,
      "val" : "0.33 EV"
   },
   "ExposureTime" : {
      "type" : 5,
      "val" : "1/50 sec."
   },
   "Model" : {
      "type" : 2,
      "val" : "Canon EOS 600D"
   },
   "ISOSpeedRatings" : {
      "type" : 3,
      "val" : "3200"
   },
   "ResolutionUnit" : {
      "type" : 3,
      "val" : "英寸"
   },
   ...其余略...
}
\end{verbatim}
  \label{exif}
\end{sample}

\clearpage

\section{生成缩略图}

\qpar{在图片下载URL后附加\qqbox{imageView}指示符(区分大小写)并指定缩略规格,可以生成该图片的缩略图。}
\qpar{调用接口:}
\begin{lstlisting}[basicstyle=\ttfamily\footnotesize]
<URL>?imageView/<Mode>/w/<Width>/h/<Height>/q/<Quality>/format/<Format>
\end{lstlisting}

%\begin{verbatim}
%<URL>?imageView/<Mode>/w/<Width>/h/<Height>/q/<Quality>/format/<Format>
%     |________||_____||________||_________||__________||______________|
%          |       |        |         |           |            |
% 指示符 --+       |        |         |           |            |
%                  |        |         |           |            |
% 缩略模式 --------+        |         |           |            |
%                           |         |           |            |
% 缩略宽度(像素) ---------+         |           |            |
%                                     |           |            |
% 缩略高度(像素) -------------------+           |            |
%                                                 |            |
% 缩略质量(取值1-100) --------------------------+            |
%                                                              |
% 图片格式(支持jpg、gif、webp等) ----------------------------+
%\end{verbatim}

\qpar{参数说明:}
\qtable{
\def\arraystretch{1.5}
\begin{tabular}{|l|l|l|}
\hline
名称 & 说明 & 备注\\
\hline
\textless Mode\textgreater & 处理模式 & \\
\hline
\textless Width\textgreater & 缩略宽度,单位:像素(px) & \\
\hline
\textless Height\textgreater & 缩略高度,单位:像素(px) & \\
\hline
\textless Quality\textgreater & 图片质量,取值范围1-100 & 可选,缺省为85 \\
\hline
\textless Format\textgreater & 图片格式,支持jpg、gif、png、webp等 & 可选,缺省为原图格式 \\
\hline
\end{tabular}
}

\qpar{其中\textless mode\textgreater分为如下4种情况:}
\qtable{
\def\arraystretch{1.5}
\begin{tabular}{|l|p{22em}|l|}
\hline
模式 & 说明 & 备注 \\
\hline
\textless mode\textgreater=1 & 同时指定宽度和高度,等比裁剪原图正中部分并缩小为\textless width\textgreater x\textless height\textgreater 大小的缩略图。& 参看\hyperref[imageView-1-200x200]{示例\ref*{imageView-1-200x200}} \\
\hline
\textless mode\textgreater=2 & 同时指定宽度和高度,原图缩小为不超出\textless width\textgreater x\textless height\textgreater 大小的缩略图,避免裁剪长边。& 参看\hyperref[imageView-2-200x200]{示例\ref*{imageView-2-200x200}} \\
\hline
\textless mode\textgreater=2 & 仅指定宽度,高度等比缩小。 & 参看\hyperref[imageView-2-200x200]{示例\ref*{imageView-2-200x200}} \\
\hline
\textless mode\textgreater=2 & 仅指定高度,宽度等比缩小。 & 参看\hyperref[imageView-2-x200]{示例\ref*{imageView-2-x200}} \\
\hline
\end{tabular}
}

\clearpage

\begin{sample}
  \caption{裁剪正中部分,等比缩小生成200x200缩略图}
  
    \qpar{\qurl{http://qiniuphotos.qiniudn.com/gogopher.jpg}}
    \qpar{$\Rightarrow$}
    \qpar{\qurl{http://qiniuphotos.qiniudn.com/gogopher.jpg?imageView/1/w/200/h/200}}

    \begin{center}
      \includegraphics[scale=0.65]{../pics/gogopher_imageView_1_200x200.jpg}
    \end{center}

  \label{imageView-1-200x200}
\end{sample}

\begin{sample}
  \caption{宽度固定为200px,高度等比缩小,生成200x133缩略图}
    \qpar{\qurl{http://qiniuphotos.qiniudn.com/gogopher.jpg}}
    \qpar{$\Rightarrow$}
    \qpar{\qurl{http://qiniuphotos.qiniudn.com/gogopher.jpg?imageView/2/w/200/h/200}}
    \qpar{\qurl{http://qiniuphotos.qiniudn.com/gogopher.jpg?imageView/2/w/200}}

    \begin{center}
    \includegraphics[scale=0.65]{../pics/gogopher_imageView_2_200x200.jpg}
    \end{center}
  \label{imageView-2-200x200}
\end{sample}

\begin{sample}
  \caption{高度固定为200px,宽度等比缩小,生成300x200缩略图}
    \qpar{\qurl{http://qiniuphotos.qiniudn.com/gogopher.jpg}}
    \qpar{$\Rightarrow$}
    \qpar{\qurl{http://qiniuphotos.qiniudn.com/gogopher.jpg?imageView/2/h/200}}

    \begin{center}
    \includegraphics[scale=0.65]{../pics/gogopher_imageView_2_x200.jpg}
    \end{center}
  \label{imageView-2-x200}
\end{sample}

\clearpage

\section{高级图片处理}

\qpar{除了能方便地生成缩略图外,七牛云处理服务还提供其它高级图片处理功能,包括缩略、裁剪、旋转等等。}
\qpar{调用接口:}
\begin{lstlisting}[basicstyle=\ttfamily\footnotesize]
<URL>?imageMogr/v2/auto-orient
                  /thumbnail/<ImageSize>
                  /gravity/<GravityType>
                  /crop/<ImageSizeAndOffset>
                  /quality/<Quality>
                  /rotate/<Degree>
                  /format/<Format>
\end{lstlisting}

\qpar{参数说明:}
\qtable{
\def\arraystretch{1.5}
\begin{tabular}{|l|p{18em}|p{10em}|}
\hline
名称 & 说明 & 备注\\
\hline
\textless ImageSize\textgreater & 缩略图规格,参见规格说明表 & 可选,缺省为不缩略 \\
\hline
\textless GravityType\textgreater & 位置偏移,只会影响其后的裁剪偏移(Offset),可选值参看位置偏移取值表 & 可选,缺省为左上角(NorthWest) \\
\hline
\textless ImageSizeAndOffset\textgreater & 裁剪规格,参见规格说明表 & 可选,缺省为不裁剪 \\
\hline
\textless Quality\textgreater & 图片质量,取值范围1-100 & 可选,缺省为85 \\
\hline
\textless Degree\textgreater & 旋转角度,取值范围1-360 & 可选,缺省为不旋转 \\
\hline
\textless Format\textgreater & 图片格式,支持jpg、gif、png、webp等 & 可选,缺省为原图格式 \\
\hline
\end{tabular}
}

\qpar{位置偏移取值表:}
\qtable{
\begin{tabular}[t]{|l|c|r|}
\hline
\parbox[t][6em]{6em}{NorthWest} & \parbox[t][6em]{6em}{\makebox[6em][c]{North}} & \parbox[t][6em]{6em}{\makebox[6em][r]{NorthEast}} \\
\hline
\parbox[c][7em]{6em}{West} & \parbox[c][7em]{6em}{\makebox[6em][c]{Center}} & \parbox[c][7em]{6em}{\makebox[6em][r]{East}} \\
\hline
\parbox[b][6em]{6em}{SouthWest} & \parbox[b][6em]{6em}{\makebox[6em][c]{South}} & \parbox[b][6em]{6em}{\makebox[6em][r]{SouthEast}} \\
\hline
\end{tabular}
}

\clearpage

\qpar{规格说明表:}
\qtable{
\begin{tabular}[t]{|l|p{20em}|p{3em}|p{3em}|}
\hline
规格 & 说明 & 缩略图 & 裁剪 \\
\hline
\textless Scale\textgreater\% & 基于原图大小,按指定百分比缩放。取值范围0-1000 & 适用 & 适用 \\
\hline
\textless Scale\textgreater\%x & 以百分比形式指定缩略图宽度,高度等比缩放。取值范围0-1000 & 适用 & 适用 \\
\hline
x\textless Scale\textgreater\% & 以百分比形式指定缩略图高度,宽度等比缩放。取值范围0-1000 & 适用 & 适用 \\
\hline
\textless Width\textgreater & 指定缩略图宽度,高度等比缩放。取值范围0-10000 & 适用 & 适用 \\
\hline
x\textless Height\textgreater & 指定缩略图高度,宽度等比缩放。取值范围0-10000 & 适用 & 适用 \\
\hline
\textless Width\textgreater x\textless Height\textgreater & 限定长边,短边自适应缩放,将缩略图限制在指定宽高矩形内。取值范围0-10000 & 适用 & 适用 \\
\hline
\textless Width\textgreater x\textless Height\textgreater \^ & 限定短边,长边自适应缩放,缩略图会延伸至指定宽高矩形外。取值范围0-10000 & 适用 & 适用 \\
\hline
\textless Width\textgreater x\textless Height\textgreater ! & 限定缩略图宽高值,忽略原图宽高比例,按照指定宽高值强行缩略,可能导致缩略图变形。取值范围0-10000 & 适用 & 适用 \\
\hline
\textless Width\textgreater x\textless Height\textgreater \textgreater & 当原图尺寸大于给定的宽度或高度时,按照给定宽高值缩小。取值范围0-10000 & 适用 & 适用 \\
\hline
\textless Width\textgreater x\textless Height\textgreater \textless & 当原图尺寸小于给定的宽度或高度时,按照给定宽高值放大。取值范围0-10000 & 适用 & 适用 \\
\hline
\textless Area\textgreater @ & 按原图高宽比例等比缩放,缩放后的像素数量不超过指定值。取值范围0-10$^8$ & 适用 & 适用 \\
\hline
\end{tabular}
}

\end{CJK*}

\end{document}
