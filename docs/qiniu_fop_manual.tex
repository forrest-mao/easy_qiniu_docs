\documentclass[11pt, oneside]{book}
\usepackage{geometry}
\geometry{a4paper}

\usepackage{CJKutf8}

\usepackage{graphicx}
\usepackage{indentfirst}
\usepackage[colorlinks,urlcolor=blue]{hyperref}

\usepackage{listings}
\lstset{basicstyle=\ttfamily}

\newcommand{\qpar}[1]{
\vspace{0.25em}
\noindent
#1\par
\vspace{0.25em}
}

\newcommand{\qurl}[1]{\url{#1}}

\newcommand{\qqbox}[1]{\hspace{0.2em}\fbox{#1}\hspace{0.2em}}

\newcommand{\qtable}[1]{\vspace{0.5em}#1\vspace{0.5em}}


\title{七牛云存储 \\ 云处理使用指南}
\author{梁涛 @七牛云存储}
\date{2013-08-19}

\begin{document}

\begin{CJK*}{UTF8}{song}
\maketitle

\chapter{图片处理}

\section{获取图片信息}

\subsection{获取基本信息}

\qpar{在图片下载URL后附加\qqbox{imageInfo}指示符(区分大小写),可以获取JSON格式的图片基本信息,包括图片格式、长宽值(单位:像素)、色彩模型。}
\qpar{缩略图不支持该方法。}

\begin{quote}
示例URL(点击查看):\par
\qurl{http://qiniuphotos.qiniudn.com/gogopher.jpg}\par
$\Rightarrow$\par
\qurl{http://qiniuphotos.qiniudn.com/gogopher.jpg?imageInfo}\par
\end{quote}

\begin{quote}
返回结果:
\begin{verbatim}
{
   "format":     "jpeg",
   "width":      640,     // 像素
   "height":     427,     // 像素
   "colorModel": "ycbcr"
}
\end{verbatim}
\end{quote}

\subsection{获取EXIF信息}

\qpar{EXIF(EXchangeable Image File Format),是专门为数码相机的照片设定的可交换图像文件格式,
详见\href{http://zh.wikipedia.org/wiki/EXIF}{维基百科}。}
\qpar{获取方式为在图片下载URL后附加\qqbox{exif}指示符(区分大小写)。}
\qpar{缩略图不支持该方法。}

\begin{quote}
示例URL(点击查看):\par
\qurl{http://qiniuphotos.qiniudn.com/gogopher.jpg}\par
$\Rightarrow$\par
\qurl{http://qiniuphotos.qiniudn.com/gogopher.jpg?exif}\par
\end{quote}

\begin{quote}
返回结果:
\begin{verbatim}
{
   "DateTime" : {
      "type" : 2,
      "val" : "2011:11:19 17:09:23"
   },
   "ExposureBiasValue" : {
      "type" : 10,
      "val" : "0.33 EV"
   },
   "ExposureTime" : {
      "type" : 5,
      "val" : "1/50 sec."
   },
   "Model" : {
      "type" : 2,
      "val" : "Canon EOS 600D"
   },
   "ISOSpeedRatings" : {
      "type" : 3,
      "val" : "3200"
   },
   "ResolutionUnit" : {
      "type" : 3,
      "val" : "英寸"
   },
   "MeteringMode" : {
      "type" : 3,
      "val" : "样式"
   },
   ...略...
}
\end{verbatim}
\end{quote}

\clearpage

\section{生成缩略图}

\qpar{在图片下载URL后附加\qqbox{imageView}指示符与缩略规格,可以生成该图片的缩略图。}
\qpar{调用接口:}
\begin{lstlisting}[basicstyle=\ttfamily\footnotesize]
<URL>?imageView/<mode>/w/<width>/h/<height>/q/<quality>/format/<format>
\end{lstlisting}

\qpar{参数说明:}
\qtable{
\def\arraystretch{1.5}
\begin{tabular}{|l|l|}
\hline
名称 & 说明 \\
\hline
\textless mode\textgreater & 缩略处理模式 \\
\hline
\textless width\textgreater & 缩略图宽度,单位:像素(px) \\
\hline
\textless height\textgreater & 缩略图高度,单位:像素(px) \\
\hline
\textless quality\textgreater & 缩略图质量,取值范围1-100 \\
\hline
\textless format\textgreater & 缩略图格式,支持jpg、gif、png、webp等 \\
\hline
\end{tabular}
}

\end{CJK*}

\end{document}
